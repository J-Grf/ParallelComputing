\section{Implementation and Validation}

\subsection{Convergence Criterion}
The unsteady solution of the heat equation converges to a stead-state for $t\rightarrow \infty$. In order to break out of the time loop a convergence criterion has to be defined, which states the change of the temperature between two subsequent timesteps:
\begin{equation}
	\epsilon_{RMS} = \sqrt{\frac{1}{nn} \sum_{i=1}^{nn} \Big( T_i^{n+1} - T_i^n \Big)^2} \leq 10^{-7}
\end{equation}
with the root-mean-square error $\epsilon_{RMS}$ and the number of nodes $nn \in \eta \backslash \eta_D$. If $\epsilon_{RMS}$ is lower than a predefined error ($10^{-7}$), the simulation is declared as converged and is finished.

\subsection{Serial Code}

The calculation of the element matrices is performed as presented in \refEq{eq:M} to \refEq{eq:F}. The corresponding code is illustrated in \refCode{Code:Serial1}. The values of the shape functions and their derivatives, as well as the determinate of the jacobian is precomputed for each element node. The matrices M, K and F are computed at element level. After performing the mass lumping procedure the element level matrices must be hard copied to the corresponding triangular element.
Then the Dirichlet boundary condition \refEq{eq:Dirichlet} is applied at the domain boundary nodes. The global mass for each node is accumulated. 

\begin{lstlisting}[caption={\label{Code:Serial1} Calculation of Element Matrices}]
// First, fill M, K, F matrices with zero for the current element
(...)

// Now, calculate the M, K, F matrices
for(int p=0; p<nGQP; p++)
{   
	(...)
	for(int i=0; i<nen; i++)
	{
		for(int j=0; j<nen; j++)
		{
			// Consistent mass matrix
			M[i][j] = M[i][j] +
			mesh->getME(p)->getS(i) * mesh->getME(p)->getS(j) *
			mesh->getElem(e)->getDetJ(p) * mesh->getME(p)->getWeight();
			// Stiffness matrix
			K[i][j] = K[i][j] +
			D * mesh->getElem(e)->getDetJ(p) * mesh->getME(p)->getWeight() *
			(mesh->getElem(e)->getDSdX(p,i) * mesh->getElem(e)->getDSdX(p,j) +
			mesh->getElem(e)->getDSdY(p,i) * mesh->getElem(e)->getDSdY(p,j));
		}
		// Forcing matrix
		F[i] = F[i] + factor_F * mesh->getME(p)->getS(i);
		
	}
}

//Calculation of total mass and the total diagonal mass and perform mass lumping
(...)

//Total mass at each node is accumulated on local node structure:
for(int i=0; i<nen; i++)
{
	node = mesh->getElem(e)->getConn(i);
	mesh->getNode(node)->addMass(M[i][i]);
}

// At this point we have the necessary K, M, F matrices as a member of femSolver 
// object.
// They must be hard copied to the corresponding triElement variables.
for(int i=0; i<nen; i++)
{
	node = mesh->getElem(e)->getConn(i);
	mesh->getElem(e)->setF(i, F[i]);
	mesh->getElem(e)->setM(i, M[i][i]);
	for(int j=0; j<nen; j++)
	{
		mesh->getElem(e)->setK(i,j,K[i][j]);
	}
}
\end{lstlisting}

After the previous computations the explicit solver is entered. The outer loop is an iteration loop in which the convergence criterion is checked at the end (\refCode{Code:Serial2}). If convergence is reached, the code breaks out of the loop. After clearing all entries of the RHS storage MTnew, the second inner loop iterates over all elements, evaluating the RHS at element level (line 9 to 22 in \refCode{Code:Serial2}). This is followed by the computation of the new temperature, using the inverted mass matrix, which inverse corresponds to it's reciprocal as the mass lumping algorithm was previously applied to diagonalize the mass matrix. Subsequently, the global RMS error can be computed and the convergence criterion is checked.

\begin{lstlisting}[caption={\label{Code:Serial2} Explicit Solver}]
	for (int iter=0; iter<nIter; iter++)
	{
		// clear RHS MTnew
		(...)
		// Evaluate right hand side at element level
		for(int e=0; e<ne; e++)
		{
			(...)
			for(int i=0; i<nen; i++)
			{
				TL[i] = T[elem->getConn(i)];
			}
			
			MTnewL[0] = M[0]*TL[0] + dT*(F[0]-(K[0]*TL[0]+K[1]*TL[1]+K[2]*TL[2]));
			MTnewL[1] = M[1]*TL[1] + dT*(F[1]-(K[3]*TL[0]+K[4]*TL[1]+K[5]*TL[2]));
			MTnewL[2] = M[2]*TL[2] + dT*(F[2]-(K[6]*TL[0]+K[7]*TL[1]+K[8]*TL[2]));
			
			// RHS is accumulated at local nodes
			MTnew[elem->getConn(0)] += MTnewL[0];
			MTnew[elem->getConn(1)] += MTnewL[1];
			MTnew[elem->getConn(2)] += MTnewL[2];
		}
		
		// Evaluate the new temperature on each node on partition level
		partialL2error = 0.0;
		globalL2error = 0.0;
		for(int i=0; i<nn; i++)
		{
			pNode = mesh->getNode(i);
			if(pNode->getBCtype() != 1)
			{
				massTmp = massG[i];
				MT = MTnew[i];
				Tnew = MT/massTmp;
				partialL2error += pow(T[i]-Tnew,2);
				T[i] = Tnew;
				MTnew[i] = 0;
			}
		}
		globalL2error = sqrt(partialL2error/this->nnSolved);
		globalL2error = globalL2error / initialL2error;
\end{lstlisting}

\subsection{Parallelization with OpenMP \label{sec:POpenMP}}

The two most time consuming loops, which are the computation of the RHS and the computation of the global RMS error are to be parallelized on loop level with a shared memory approach. This means multiple threads share the work of the comutation inside a loop, and have access to the same memory address space.

Therefore OpenMP directives are used. Two different approaches are implemented, denoted with "A)" and "B)" which are assessed in \refSec{sec:OpenMPAB}. The corresponding directives are presented in \refCode{Code:OpenMP}.

For approach "A)" MTnew is declared as firstprivate upon entry in the parallel region. This ensures, that the entries MTnew are initialized at the entry of the parallel section, which means MTnew is copied from the parent thread to the child threads. For approach "B)" no initalialization of MTnew is performed on the child threads, as it is a shared array among all threads. Both approaches declare the element object elem, the mass matrix M, the stiffness matrix K, the source vector F, the local temperature, the node loop index i and the local RHS MTnewL as private. This means all these objects are private (local) to the thread and undefined upon entry to the region. For approach A), critical region are used for the accumulation of the RHS at the local nodes and the computation of the partial L2 errors. 

A critical region is to be executed by only one thread at a time. This is important to prevent data races. A data race means that multiple threads try to access and manipulate the same variable / memory address at the same time. This might lead to unpredictable behavior, which can result in incorrect computations. 

For approach B), reduction operations are used. The reduction operation for the computation of the RHS at element level for each element node, allows each thread to first compute local sums of the each MTnew entry in parallel and then performs a serial summation of all local sums to obtain the value of an MTnew entry. But this serial summation is only computed once, compared to the critical region. The same procedure applies also for the computation of the partial L2 error. 

In this context different scheduling options, with different chunk sizes can be compared (\refSec{sec:Scheduling}). Explaining the different scheduling options briefly, static scheduling accounts for a static chunk size, meaning a chunk is consisting of a constant number of iterations. These chunks are then distributed in a round-robin fashion among the threads. "Dynamic" scheduling however allows the chunk size to be allocated dynamically, based on the load of a thread. Guided scheduling works similar to "dynamic" scheduling, but with a chunk size that is constantly decreasing, based on the unassigned iterations left. The scheduling option "auto" leaves this decision to the compiler.
\newpage

\begin{lstlisting}[caption={\label{Code:OpenMP} Parallelization of two most time consuming loops with implementation A and B}]
	for (int iter=0; iter<nIter; iter++)
	{
		// clear RHS MTnew
		A) #pragma omp parallel for
		B) #pragma omp parallel for
		for(i=0; i<nn; i++){
			MTnew[i] = 0;
		}
		A) #pragma omp parallel firstprivate(MTnew)
		B) #pragma omp parallel
		{
			// Evaluate right hand side at element level
			A) #pragma omp for private(elem, M, F, K, TL, i, MTnewL)
			B) #pragma omp for private(elem, M, F, K, TL, i, MTnewL) 
			B) reduction(+: MTnew[0:nn]) schedule(dynamic,512)
			for(int e=0; e<ne; e++)
			{
				elem = mesh->getElem(e);
				M = elem->getMptr();
				F = elem->getFptr();
				K = elem->getKptr();
				for(i=0; i<nen; i++)
				{
					TL[i] = T[elem->getConn(i)];
				}
				
				MTnewL[0] = M[0]*TL[0] + dT*(F[0]-(K[0]*TL[0]+...));
				MTnewL[1] = M[1]*TL[1] + dT*(F[1]-(K[3]*TL[0]+...));
				MTnewL[2] = M[2]*TL[2] + dT*(F[2]-(K[6]*TL[0]+...));
				
				
				// RHS is accumulated at local nodes
				A) #pragma omp critical
				MTnew[elem->getConn(0)] += MTnewL[0];
				A) #pragma omp critical
				MTnew[elem->getConn(1)] += MTnewL[1];
				A) #pragma omp critical
				MTnew[elem->getConn(2)] += MTnewL[2];
			}
			// Evaluate the new temperature on each node on partition level
			partialL2error = 0.0;
			globalL2error = 0.0;
			A) #pragma omp for private(pNode, massTmp, MT, Tnew)
			B) #pragma omp for private(pNode, massTmp, MT, Tnew) 
			B) reduction(+:partialL2error) schedule(dynamic,512)
			for(int i=0; i<nn; i++)
			{
				pNode = mesh->getNode(i);
				if(pNode->getBCtype() != 1)
				{
					massTmp = massG[i];
					MT = MTnew[i];
					Tnew = MT/massTmp;
					A) #pragma omp critical
					partialL2error += pow(T[i]-Tnew,2);
					T[i] = Tnew;
				}
			}
		}
		
		globalL2error = sqrt(partialL2error/nn);
\end{lstlisting}

\subsection{Parallelization with MPI}

\begin{lstlisting}[caption={\label{Code:MPI1} Determination of number elements on current rank and number nodes in current}]
	// Determine nec, mec
	nec = (ne-1)/npes + 1;
	mec = nec;
	if ((mype+1)*mec > ne)
	nec = ne - mype*mec;
	if (nec < 0)
	nec = 0;
	
	// Determine nnc, mnc
	nnc = (nn-1)/npes + 1;
	mnc = nnc;
	if ((mype+1)*mnc > nn)
	nnc = nn - mype*mnc;
	if (nnc < 0)
	nnc = 0;
\end{lstlisting}

\begin{lstlisting}[caption={\label{Code:MPI2} Read file for every rank with specific offset}]
	(...)
	offset = mype*nsd*mnc*sizeof(double);
	MPI_Type_contiguous(nnc*nsd, MPI_DOUBLE, &mxyzftype);
	MPI_Type_commit(&mxyzftype);
	MPI_File_open(MPI_COMM_WORLD, writable, MPI_MODE_RDONLY, 
	MPI_INFO_NULL, &fileptr);
	MPI_File_set_view(fileptr, offset, MPI_DOUBLE, mxyzftype, 
	"native", MPI_INFO_NULL);
	readStream = new char [nsd*nnc*sizeof(double)];
	MPI_File_read(fileptr,readStream, nsd*nnc, MPI_DOUBLE, &status);
	swapBytes(readStream, nsd*nnc, sizeof(double));
	for(int i=0; i<nnc; i++)
	{
		node[i].setX(*((double*)readStream + nsd*i));
		node[i].setY(*((double*)readStream + nsd*i+1));
		xyz[i*nsd+xsd] = *((double*)readStream + nsd*i);
		xyz[i*nsd+ysd] = *((double*)readStream + nsd*i+1);
	}
	if (mype==0) cout << "> File read complete: " << dummy << endl;
	
	MPI_File_close(&fileptr);
	MPI_Barrier(MPI_COMM_WORLD);
	
\end{lstlisting}

- explain mapping from local to global node list
- calculate element matrices is performed on every core

\begin{lstlisting}[caption={\label{Code:MPI3} Explicit solver for MPI parallelization}]
	MPI_Win winMTnew;
	MPI_Win winTG;
	
	MPI_Win_create(MTnewG, nnc*sizeof(double), sizeof(double), MPI_INFO_NULL, 
	MPI_COMM_WORLD, &winMTnew);
	MPI_Win_create(TG, nnc*sizeof(double), sizeof(double), MPI_INFO_NULL, 
	MPI_COMM_WORLD, &winTG);
	MPI_Win_fence(0, winMTnew);
	MPI_Win_fence(0, winTG);
	
	localizeTemperature(winTG);
	
	for (int iter=0; iter<nIter; iter++)
	{
		
		// Similar to serial Code, RHS is evaluated at element level
		// and accumulated at local nodes
		(...)
		 
		// local node level MTnew is transferred to partition node level (MTnewL to MTnewG)
		accumulateMTnew(winMTnew);
		
		// Evaluate the new temperature on each node on partition level
		for(int i=0; i<nnc; i++)
		{
			pNode = mesh->getNode(i);
			(...)
		}
		
		// Transfer new temperatures from partition to local node level (from TG to TL)
		localizeTemperature(winTG);
		
		MPI_Allreduce(&partialL2error, &globalL2error, 1, MPI_DOUBLE, MPI_SUM, 
		MPI_COMM_WORLD);
		
		globalL2error = sqrt(globalL2error/(double)this->nnSolved);
		
		// Output of iterations and globalL2error on parent processor
	
	MPI_Win_free(&winMTnew);
	MPI_Win_free(&winTG);
	
	partialsumT = 0.0;
	for (int i = 0; i < nnc; i++)
	{
		partialsumT += TG[i];
	}
	MPI_Allreduce(&partialsumT, &sumT, 1, MPI_DOUBLE, MPI_SUM, MPI_COMM_WORLD);
\end{lstlisting}

-accumulate mass
-localize temperature
-accumulate MTNew

\subsection{R1 a) Validation}
For the validation of the FEM solver, the solution of the coarse mesh is compared to the analytical solution in \refFig{fig::TemperatureDist}. The numeric solution is in good agreement with the analytical solution. Slight deviations are observed at the disk's center where the temperature distribution has it's global maximum. Accounting for large changes in the temperature gradient at the center of the disk. 

\begin{figure}[!htbp]
	\centering
	%\hspace*{0.8cm}
	\leavevmode
	\resizebox{0.8\width}{!}{%% Creator: Matplotlib, PGF backend
%%
%% To include the figure in your LaTeX document, write
%%   \input{<filename>.pgf}
%%
%% Make sure the required packages are loaded in your preamble
%%   \usepackage{pgf}
%%
%% and, on pdftex
%%   \usepackage[utf8]{inputenc}\DeclareUnicodeCharacter{2212}{-}
%%
%% or, on luatex and xetex
%%   \usepackage{unicode-math}
%%
%% Figures using additional raster images can only be included by \input if
%% they are in the same directory as the main LaTeX file. For loading figures
%% from other directories you can use the `import` package
%%   \usepackage{import}
%%
%% and then include the figures with
%%   \import{<path to file>}{<filename>.pgf}
%%
%% Matplotlib used the following preamble
%%
\begingroup%
\makeatletter%
\begin{pgfpicture}%
\pgfpathrectangle{\pgfpointorigin}{\pgfqpoint{6.565064in}{4.725328in}}%
\pgfusepath{use as bounding box, clip}%
\begin{pgfscope}%
\pgfsetbuttcap%
\pgfsetmiterjoin%
\definecolor{currentfill}{rgb}{1.000000,1.000000,1.000000}%
\pgfsetfillcolor{currentfill}%
\pgfsetlinewidth{0.000000pt}%
\definecolor{currentstroke}{rgb}{1.000000,1.000000,1.000000}%
\pgfsetstrokecolor{currentstroke}%
\pgfsetdash{}{0pt}%
\pgfpathmoveto{\pgfqpoint{0.000000in}{0.000000in}}%
\pgfpathlineto{\pgfqpoint{6.565064in}{0.000000in}}%
\pgfpathlineto{\pgfqpoint{6.565064in}{4.725328in}}%
\pgfpathlineto{\pgfqpoint{0.000000in}{4.725328in}}%
\pgfpathclose%
\pgfusepath{fill}%
\end{pgfscope}%
\begin{pgfscope}%
\pgfsetbuttcap%
\pgfsetmiterjoin%
\definecolor{currentfill}{rgb}{1.000000,1.000000,1.000000}%
\pgfsetfillcolor{currentfill}%
\pgfsetlinewidth{0.000000pt}%
\definecolor{currentstroke}{rgb}{0.000000,0.000000,0.000000}%
\pgfsetstrokecolor{currentstroke}%
\pgfsetstrokeopacity{0.000000}%
\pgfsetdash{}{0pt}%
\pgfpathmoveto{\pgfqpoint{1.081443in}{0.626234in}}%
\pgfpathlineto{\pgfqpoint{6.256884in}{0.626234in}}%
\pgfpathlineto{\pgfqpoint{6.256884in}{4.575328in}}%
\pgfpathlineto{\pgfqpoint{1.081443in}{4.575328in}}%
\pgfpathclose%
\pgfusepath{fill}%
\end{pgfscope}%
\begin{pgfscope}%
\pgfpathrectangle{\pgfqpoint{1.081443in}{0.626234in}}{\pgfqpoint{5.175441in}{3.949094in}}%
\pgfusepath{clip}%
\pgfsetrectcap%
\pgfsetroundjoin%
\pgfsetlinewidth{0.803000pt}%
\definecolor{currentstroke}{rgb}{0.690196,0.690196,0.690196}%
\pgfsetstrokecolor{currentstroke}%
\pgfsetdash{}{0pt}%
\pgfpathmoveto{\pgfqpoint{1.081443in}{0.626234in}}%
\pgfpathlineto{\pgfqpoint{1.081443in}{4.575328in}}%
\pgfusepath{stroke}%
\end{pgfscope}%
\begin{pgfscope}%
\pgfsetbuttcap%
\pgfsetroundjoin%
\definecolor{currentfill}{rgb}{0.000000,0.000000,0.000000}%
\pgfsetfillcolor{currentfill}%
\pgfsetlinewidth{0.803000pt}%
\definecolor{currentstroke}{rgb}{0.000000,0.000000,0.000000}%
\pgfsetstrokecolor{currentstroke}%
\pgfsetdash{}{0pt}%
\pgfsys@defobject{currentmarker}{\pgfqpoint{0.000000in}{-0.048611in}}{\pgfqpoint{0.000000in}{0.000000in}}{%
\pgfpathmoveto{\pgfqpoint{0.000000in}{0.000000in}}%
\pgfpathlineto{\pgfqpoint{0.000000in}{-0.048611in}}%
\pgfusepath{stroke,fill}%
}%
\begin{pgfscope}%
\pgfsys@transformshift{1.081443in}{0.626234in}%
\pgfsys@useobject{currentmarker}{}%
\end{pgfscope}%
\end{pgfscope}%
\begin{pgfscope}%
\definecolor{textcolor}{rgb}{0.000000,0.000000,0.000000}%
\pgfsetstrokecolor{textcolor}%
\pgfsetfillcolor{textcolor}%
\pgftext[x=1.081443in,y=0.529012in,,top]{\color{textcolor}\rmfamily\fontsize{10.000000}{12.000000}\selectfont \(\displaystyle {-0.100}\)}%
\end{pgfscope}%
\begin{pgfscope}%
\pgfpathrectangle{\pgfqpoint{1.081443in}{0.626234in}}{\pgfqpoint{5.175441in}{3.949094in}}%
\pgfusepath{clip}%
\pgfsetrectcap%
\pgfsetroundjoin%
\pgfsetlinewidth{0.803000pt}%
\definecolor{currentstroke}{rgb}{0.690196,0.690196,0.690196}%
\pgfsetstrokecolor{currentstroke}%
\pgfsetdash{}{0pt}%
\pgfpathmoveto{\pgfqpoint{1.728373in}{0.626234in}}%
\pgfpathlineto{\pgfqpoint{1.728373in}{4.575328in}}%
\pgfusepath{stroke}%
\end{pgfscope}%
\begin{pgfscope}%
\pgfsetbuttcap%
\pgfsetroundjoin%
\definecolor{currentfill}{rgb}{0.000000,0.000000,0.000000}%
\pgfsetfillcolor{currentfill}%
\pgfsetlinewidth{0.803000pt}%
\definecolor{currentstroke}{rgb}{0.000000,0.000000,0.000000}%
\pgfsetstrokecolor{currentstroke}%
\pgfsetdash{}{0pt}%
\pgfsys@defobject{currentmarker}{\pgfqpoint{0.000000in}{-0.048611in}}{\pgfqpoint{0.000000in}{0.000000in}}{%
\pgfpathmoveto{\pgfqpoint{0.000000in}{0.000000in}}%
\pgfpathlineto{\pgfqpoint{0.000000in}{-0.048611in}}%
\pgfusepath{stroke,fill}%
}%
\begin{pgfscope}%
\pgfsys@transformshift{1.728373in}{0.626234in}%
\pgfsys@useobject{currentmarker}{}%
\end{pgfscope}%
\end{pgfscope}%
\begin{pgfscope}%
\definecolor{textcolor}{rgb}{0.000000,0.000000,0.000000}%
\pgfsetstrokecolor{textcolor}%
\pgfsetfillcolor{textcolor}%
\pgftext[x=1.728373in,y=0.529012in,,top]{\color{textcolor}\rmfamily\fontsize{10.000000}{12.000000}\selectfont \(\displaystyle {-0.075}\)}%
\end{pgfscope}%
\begin{pgfscope}%
\pgfpathrectangle{\pgfqpoint{1.081443in}{0.626234in}}{\pgfqpoint{5.175441in}{3.949094in}}%
\pgfusepath{clip}%
\pgfsetrectcap%
\pgfsetroundjoin%
\pgfsetlinewidth{0.803000pt}%
\definecolor{currentstroke}{rgb}{0.690196,0.690196,0.690196}%
\pgfsetstrokecolor{currentstroke}%
\pgfsetdash{}{0pt}%
\pgfpathmoveto{\pgfqpoint{2.375303in}{0.626234in}}%
\pgfpathlineto{\pgfqpoint{2.375303in}{4.575328in}}%
\pgfusepath{stroke}%
\end{pgfscope}%
\begin{pgfscope}%
\pgfsetbuttcap%
\pgfsetroundjoin%
\definecolor{currentfill}{rgb}{0.000000,0.000000,0.000000}%
\pgfsetfillcolor{currentfill}%
\pgfsetlinewidth{0.803000pt}%
\definecolor{currentstroke}{rgb}{0.000000,0.000000,0.000000}%
\pgfsetstrokecolor{currentstroke}%
\pgfsetdash{}{0pt}%
\pgfsys@defobject{currentmarker}{\pgfqpoint{0.000000in}{-0.048611in}}{\pgfqpoint{0.000000in}{0.000000in}}{%
\pgfpathmoveto{\pgfqpoint{0.000000in}{0.000000in}}%
\pgfpathlineto{\pgfqpoint{0.000000in}{-0.048611in}}%
\pgfusepath{stroke,fill}%
}%
\begin{pgfscope}%
\pgfsys@transformshift{2.375303in}{0.626234in}%
\pgfsys@useobject{currentmarker}{}%
\end{pgfscope}%
\end{pgfscope}%
\begin{pgfscope}%
\definecolor{textcolor}{rgb}{0.000000,0.000000,0.000000}%
\pgfsetstrokecolor{textcolor}%
\pgfsetfillcolor{textcolor}%
\pgftext[x=2.375303in,y=0.529012in,,top]{\color{textcolor}\rmfamily\fontsize{10.000000}{12.000000}\selectfont \(\displaystyle {-0.050}\)}%
\end{pgfscope}%
\begin{pgfscope}%
\pgfpathrectangle{\pgfqpoint{1.081443in}{0.626234in}}{\pgfqpoint{5.175441in}{3.949094in}}%
\pgfusepath{clip}%
\pgfsetrectcap%
\pgfsetroundjoin%
\pgfsetlinewidth{0.803000pt}%
\definecolor{currentstroke}{rgb}{0.690196,0.690196,0.690196}%
\pgfsetstrokecolor{currentstroke}%
\pgfsetdash{}{0pt}%
\pgfpathmoveto{\pgfqpoint{3.022234in}{0.626234in}}%
\pgfpathlineto{\pgfqpoint{3.022234in}{4.575328in}}%
\pgfusepath{stroke}%
\end{pgfscope}%
\begin{pgfscope}%
\pgfsetbuttcap%
\pgfsetroundjoin%
\definecolor{currentfill}{rgb}{0.000000,0.000000,0.000000}%
\pgfsetfillcolor{currentfill}%
\pgfsetlinewidth{0.803000pt}%
\definecolor{currentstroke}{rgb}{0.000000,0.000000,0.000000}%
\pgfsetstrokecolor{currentstroke}%
\pgfsetdash{}{0pt}%
\pgfsys@defobject{currentmarker}{\pgfqpoint{0.000000in}{-0.048611in}}{\pgfqpoint{0.000000in}{0.000000in}}{%
\pgfpathmoveto{\pgfqpoint{0.000000in}{0.000000in}}%
\pgfpathlineto{\pgfqpoint{0.000000in}{-0.048611in}}%
\pgfusepath{stroke,fill}%
}%
\begin{pgfscope}%
\pgfsys@transformshift{3.022234in}{0.626234in}%
\pgfsys@useobject{currentmarker}{}%
\end{pgfscope}%
\end{pgfscope}%
\begin{pgfscope}%
\definecolor{textcolor}{rgb}{0.000000,0.000000,0.000000}%
\pgfsetstrokecolor{textcolor}%
\pgfsetfillcolor{textcolor}%
\pgftext[x=3.022234in,y=0.529012in,,top]{\color{textcolor}\rmfamily\fontsize{10.000000}{12.000000}\selectfont \(\displaystyle {-0.025}\)}%
\end{pgfscope}%
\begin{pgfscope}%
\pgfpathrectangle{\pgfqpoint{1.081443in}{0.626234in}}{\pgfqpoint{5.175441in}{3.949094in}}%
\pgfusepath{clip}%
\pgfsetrectcap%
\pgfsetroundjoin%
\pgfsetlinewidth{0.803000pt}%
\definecolor{currentstroke}{rgb}{0.690196,0.690196,0.690196}%
\pgfsetstrokecolor{currentstroke}%
\pgfsetdash{}{0pt}%
\pgfpathmoveto{\pgfqpoint{3.669164in}{0.626234in}}%
\pgfpathlineto{\pgfqpoint{3.669164in}{4.575328in}}%
\pgfusepath{stroke}%
\end{pgfscope}%
\begin{pgfscope}%
\pgfsetbuttcap%
\pgfsetroundjoin%
\definecolor{currentfill}{rgb}{0.000000,0.000000,0.000000}%
\pgfsetfillcolor{currentfill}%
\pgfsetlinewidth{0.803000pt}%
\definecolor{currentstroke}{rgb}{0.000000,0.000000,0.000000}%
\pgfsetstrokecolor{currentstroke}%
\pgfsetdash{}{0pt}%
\pgfsys@defobject{currentmarker}{\pgfqpoint{0.000000in}{-0.048611in}}{\pgfqpoint{0.000000in}{0.000000in}}{%
\pgfpathmoveto{\pgfqpoint{0.000000in}{0.000000in}}%
\pgfpathlineto{\pgfqpoint{0.000000in}{-0.048611in}}%
\pgfusepath{stroke,fill}%
}%
\begin{pgfscope}%
\pgfsys@transformshift{3.669164in}{0.626234in}%
\pgfsys@useobject{currentmarker}{}%
\end{pgfscope}%
\end{pgfscope}%
\begin{pgfscope}%
\definecolor{textcolor}{rgb}{0.000000,0.000000,0.000000}%
\pgfsetstrokecolor{textcolor}%
\pgfsetfillcolor{textcolor}%
\pgftext[x=3.669164in,y=0.529012in,,top]{\color{textcolor}\rmfamily\fontsize{10.000000}{12.000000}\selectfont \(\displaystyle {0.000}\)}%
\end{pgfscope}%
\begin{pgfscope}%
\pgfpathrectangle{\pgfqpoint{1.081443in}{0.626234in}}{\pgfqpoint{5.175441in}{3.949094in}}%
\pgfusepath{clip}%
\pgfsetrectcap%
\pgfsetroundjoin%
\pgfsetlinewidth{0.803000pt}%
\definecolor{currentstroke}{rgb}{0.690196,0.690196,0.690196}%
\pgfsetstrokecolor{currentstroke}%
\pgfsetdash{}{0pt}%
\pgfpathmoveto{\pgfqpoint{4.316094in}{0.626234in}}%
\pgfpathlineto{\pgfqpoint{4.316094in}{4.575328in}}%
\pgfusepath{stroke}%
\end{pgfscope}%
\begin{pgfscope}%
\pgfsetbuttcap%
\pgfsetroundjoin%
\definecolor{currentfill}{rgb}{0.000000,0.000000,0.000000}%
\pgfsetfillcolor{currentfill}%
\pgfsetlinewidth{0.803000pt}%
\definecolor{currentstroke}{rgb}{0.000000,0.000000,0.000000}%
\pgfsetstrokecolor{currentstroke}%
\pgfsetdash{}{0pt}%
\pgfsys@defobject{currentmarker}{\pgfqpoint{0.000000in}{-0.048611in}}{\pgfqpoint{0.000000in}{0.000000in}}{%
\pgfpathmoveto{\pgfqpoint{0.000000in}{0.000000in}}%
\pgfpathlineto{\pgfqpoint{0.000000in}{-0.048611in}}%
\pgfusepath{stroke,fill}%
}%
\begin{pgfscope}%
\pgfsys@transformshift{4.316094in}{0.626234in}%
\pgfsys@useobject{currentmarker}{}%
\end{pgfscope}%
\end{pgfscope}%
\begin{pgfscope}%
\definecolor{textcolor}{rgb}{0.000000,0.000000,0.000000}%
\pgfsetstrokecolor{textcolor}%
\pgfsetfillcolor{textcolor}%
\pgftext[x=4.316094in,y=0.529012in,,top]{\color{textcolor}\rmfamily\fontsize{10.000000}{12.000000}\selectfont \(\displaystyle {0.025}\)}%
\end{pgfscope}%
\begin{pgfscope}%
\pgfpathrectangle{\pgfqpoint{1.081443in}{0.626234in}}{\pgfqpoint{5.175441in}{3.949094in}}%
\pgfusepath{clip}%
\pgfsetrectcap%
\pgfsetroundjoin%
\pgfsetlinewidth{0.803000pt}%
\definecolor{currentstroke}{rgb}{0.690196,0.690196,0.690196}%
\pgfsetstrokecolor{currentstroke}%
\pgfsetdash{}{0pt}%
\pgfpathmoveto{\pgfqpoint{4.963024in}{0.626234in}}%
\pgfpathlineto{\pgfqpoint{4.963024in}{4.575328in}}%
\pgfusepath{stroke}%
\end{pgfscope}%
\begin{pgfscope}%
\pgfsetbuttcap%
\pgfsetroundjoin%
\definecolor{currentfill}{rgb}{0.000000,0.000000,0.000000}%
\pgfsetfillcolor{currentfill}%
\pgfsetlinewidth{0.803000pt}%
\definecolor{currentstroke}{rgb}{0.000000,0.000000,0.000000}%
\pgfsetstrokecolor{currentstroke}%
\pgfsetdash{}{0pt}%
\pgfsys@defobject{currentmarker}{\pgfqpoint{0.000000in}{-0.048611in}}{\pgfqpoint{0.000000in}{0.000000in}}{%
\pgfpathmoveto{\pgfqpoint{0.000000in}{0.000000in}}%
\pgfpathlineto{\pgfqpoint{0.000000in}{-0.048611in}}%
\pgfusepath{stroke,fill}%
}%
\begin{pgfscope}%
\pgfsys@transformshift{4.963024in}{0.626234in}%
\pgfsys@useobject{currentmarker}{}%
\end{pgfscope}%
\end{pgfscope}%
\begin{pgfscope}%
\definecolor{textcolor}{rgb}{0.000000,0.000000,0.000000}%
\pgfsetstrokecolor{textcolor}%
\pgfsetfillcolor{textcolor}%
\pgftext[x=4.963024in,y=0.529012in,,top]{\color{textcolor}\rmfamily\fontsize{10.000000}{12.000000}\selectfont \(\displaystyle {0.050}\)}%
\end{pgfscope}%
\begin{pgfscope}%
\pgfpathrectangle{\pgfqpoint{1.081443in}{0.626234in}}{\pgfqpoint{5.175441in}{3.949094in}}%
\pgfusepath{clip}%
\pgfsetrectcap%
\pgfsetroundjoin%
\pgfsetlinewidth{0.803000pt}%
\definecolor{currentstroke}{rgb}{0.690196,0.690196,0.690196}%
\pgfsetstrokecolor{currentstroke}%
\pgfsetdash{}{0pt}%
\pgfpathmoveto{\pgfqpoint{5.609954in}{0.626234in}}%
\pgfpathlineto{\pgfqpoint{5.609954in}{4.575328in}}%
\pgfusepath{stroke}%
\end{pgfscope}%
\begin{pgfscope}%
\pgfsetbuttcap%
\pgfsetroundjoin%
\definecolor{currentfill}{rgb}{0.000000,0.000000,0.000000}%
\pgfsetfillcolor{currentfill}%
\pgfsetlinewidth{0.803000pt}%
\definecolor{currentstroke}{rgb}{0.000000,0.000000,0.000000}%
\pgfsetstrokecolor{currentstroke}%
\pgfsetdash{}{0pt}%
\pgfsys@defobject{currentmarker}{\pgfqpoint{0.000000in}{-0.048611in}}{\pgfqpoint{0.000000in}{0.000000in}}{%
\pgfpathmoveto{\pgfqpoint{0.000000in}{0.000000in}}%
\pgfpathlineto{\pgfqpoint{0.000000in}{-0.048611in}}%
\pgfusepath{stroke,fill}%
}%
\begin{pgfscope}%
\pgfsys@transformshift{5.609954in}{0.626234in}%
\pgfsys@useobject{currentmarker}{}%
\end{pgfscope}%
\end{pgfscope}%
\begin{pgfscope}%
\definecolor{textcolor}{rgb}{0.000000,0.000000,0.000000}%
\pgfsetstrokecolor{textcolor}%
\pgfsetfillcolor{textcolor}%
\pgftext[x=5.609954in,y=0.529012in,,top]{\color{textcolor}\rmfamily\fontsize{10.000000}{12.000000}\selectfont \(\displaystyle {0.075}\)}%
\end{pgfscope}%
\begin{pgfscope}%
\pgfpathrectangle{\pgfqpoint{1.081443in}{0.626234in}}{\pgfqpoint{5.175441in}{3.949094in}}%
\pgfusepath{clip}%
\pgfsetrectcap%
\pgfsetroundjoin%
\pgfsetlinewidth{0.803000pt}%
\definecolor{currentstroke}{rgb}{0.690196,0.690196,0.690196}%
\pgfsetstrokecolor{currentstroke}%
\pgfsetdash{}{0pt}%
\pgfpathmoveto{\pgfqpoint{6.256884in}{0.626234in}}%
\pgfpathlineto{\pgfqpoint{6.256884in}{4.575328in}}%
\pgfusepath{stroke}%
\end{pgfscope}%
\begin{pgfscope}%
\pgfsetbuttcap%
\pgfsetroundjoin%
\definecolor{currentfill}{rgb}{0.000000,0.000000,0.000000}%
\pgfsetfillcolor{currentfill}%
\pgfsetlinewidth{0.803000pt}%
\definecolor{currentstroke}{rgb}{0.000000,0.000000,0.000000}%
\pgfsetstrokecolor{currentstroke}%
\pgfsetdash{}{0pt}%
\pgfsys@defobject{currentmarker}{\pgfqpoint{0.000000in}{-0.048611in}}{\pgfqpoint{0.000000in}{0.000000in}}{%
\pgfpathmoveto{\pgfqpoint{0.000000in}{0.000000in}}%
\pgfpathlineto{\pgfqpoint{0.000000in}{-0.048611in}}%
\pgfusepath{stroke,fill}%
}%
\begin{pgfscope}%
\pgfsys@transformshift{6.256884in}{0.626234in}%
\pgfsys@useobject{currentmarker}{}%
\end{pgfscope}%
\end{pgfscope}%
\begin{pgfscope}%
\definecolor{textcolor}{rgb}{0.000000,0.000000,0.000000}%
\pgfsetstrokecolor{textcolor}%
\pgfsetfillcolor{textcolor}%
\pgftext[x=6.256884in,y=0.529012in,,top]{\color{textcolor}\rmfamily\fontsize{10.000000}{12.000000}\selectfont \(\displaystyle {0.100}\)}%
\end{pgfscope}%
\begin{pgfscope}%
\definecolor{textcolor}{rgb}{0.000000,0.000000,0.000000}%
\pgfsetstrokecolor{textcolor}%
\pgfsetfillcolor{textcolor}%
\pgftext[x=3.669164in,y=0.350000in,,top]{\color{textcolor}\rmfamily\fontsize{15.000000}{18.000000}\selectfont r in [m]}%
\end{pgfscope}%
\begin{pgfscope}%
\pgfpathrectangle{\pgfqpoint{1.081443in}{0.626234in}}{\pgfqpoint{5.175441in}{3.949094in}}%
\pgfusepath{clip}%
\pgfsetrectcap%
\pgfsetroundjoin%
\pgfsetlinewidth{0.803000pt}%
\definecolor{currentstroke}{rgb}{0.690196,0.690196,0.690196}%
\pgfsetstrokecolor{currentstroke}%
\pgfsetdash{}{0pt}%
\pgfpathmoveto{\pgfqpoint{1.081443in}{0.805738in}}%
\pgfpathlineto{\pgfqpoint{6.256884in}{0.805738in}}%
\pgfusepath{stroke}%
\end{pgfscope}%
\begin{pgfscope}%
\pgfsetbuttcap%
\pgfsetroundjoin%
\definecolor{currentfill}{rgb}{0.000000,0.000000,0.000000}%
\pgfsetfillcolor{currentfill}%
\pgfsetlinewidth{0.803000pt}%
\definecolor{currentstroke}{rgb}{0.000000,0.000000,0.000000}%
\pgfsetstrokecolor{currentstroke}%
\pgfsetdash{}{0pt}%
\pgfsys@defobject{currentmarker}{\pgfqpoint{-0.048611in}{0.000000in}}{\pgfqpoint{-0.000000in}{0.000000in}}{%
\pgfpathmoveto{\pgfqpoint{-0.000000in}{0.000000in}}%
\pgfpathlineto{\pgfqpoint{-0.048611in}{0.000000in}}%
\pgfusepath{stroke,fill}%
}%
\begin{pgfscope}%
\pgfsys@transformshift{1.081443in}{0.805738in}%
\pgfsys@useobject{currentmarker}{}%
\end{pgfscope}%
\end{pgfscope}%
\begin{pgfscope}%
\definecolor{textcolor}{rgb}{0.000000,0.000000,0.000000}%
\pgfsetstrokecolor{textcolor}%
\pgfsetfillcolor{textcolor}%
\pgftext[x=0.775887in, y=0.757513in, left, base]{\color{textcolor}\rmfamily\fontsize{10.000000}{12.000000}\selectfont \(\displaystyle {500}\)}%
\end{pgfscope}%
\begin{pgfscope}%
\pgfpathrectangle{\pgfqpoint{1.081443in}{0.626234in}}{\pgfqpoint{5.175441in}{3.949094in}}%
\pgfusepath{clip}%
\pgfsetrectcap%
\pgfsetroundjoin%
\pgfsetlinewidth{0.803000pt}%
\definecolor{currentstroke}{rgb}{0.690196,0.690196,0.690196}%
\pgfsetstrokecolor{currentstroke}%
\pgfsetdash{}{0pt}%
\pgfpathmoveto{\pgfqpoint{1.081443in}{1.610609in}}%
\pgfpathlineto{\pgfqpoint{6.256884in}{1.610609in}}%
\pgfusepath{stroke}%
\end{pgfscope}%
\begin{pgfscope}%
\pgfsetbuttcap%
\pgfsetroundjoin%
\definecolor{currentfill}{rgb}{0.000000,0.000000,0.000000}%
\pgfsetfillcolor{currentfill}%
\pgfsetlinewidth{0.803000pt}%
\definecolor{currentstroke}{rgb}{0.000000,0.000000,0.000000}%
\pgfsetstrokecolor{currentstroke}%
\pgfsetdash{}{0pt}%
\pgfsys@defobject{currentmarker}{\pgfqpoint{-0.048611in}{0.000000in}}{\pgfqpoint{-0.000000in}{0.000000in}}{%
\pgfpathmoveto{\pgfqpoint{-0.000000in}{0.000000in}}%
\pgfpathlineto{\pgfqpoint{-0.048611in}{0.000000in}}%
\pgfusepath{stroke,fill}%
}%
\begin{pgfscope}%
\pgfsys@transformshift{1.081443in}{1.610609in}%
\pgfsys@useobject{currentmarker}{}%
\end{pgfscope}%
\end{pgfscope}%
\begin{pgfscope}%
\definecolor{textcolor}{rgb}{0.000000,0.000000,0.000000}%
\pgfsetstrokecolor{textcolor}%
\pgfsetfillcolor{textcolor}%
\pgftext[x=0.775887in, y=1.562383in, left, base]{\color{textcolor}\rmfamily\fontsize{10.000000}{12.000000}\selectfont \(\displaystyle {510}\)}%
\end{pgfscope}%
\begin{pgfscope}%
\pgfpathrectangle{\pgfqpoint{1.081443in}{0.626234in}}{\pgfqpoint{5.175441in}{3.949094in}}%
\pgfusepath{clip}%
\pgfsetrectcap%
\pgfsetroundjoin%
\pgfsetlinewidth{0.803000pt}%
\definecolor{currentstroke}{rgb}{0.690196,0.690196,0.690196}%
\pgfsetstrokecolor{currentstroke}%
\pgfsetdash{}{0pt}%
\pgfpathmoveto{\pgfqpoint{1.081443in}{2.415479in}}%
\pgfpathlineto{\pgfqpoint{6.256884in}{2.415479in}}%
\pgfusepath{stroke}%
\end{pgfscope}%
\begin{pgfscope}%
\pgfsetbuttcap%
\pgfsetroundjoin%
\definecolor{currentfill}{rgb}{0.000000,0.000000,0.000000}%
\pgfsetfillcolor{currentfill}%
\pgfsetlinewidth{0.803000pt}%
\definecolor{currentstroke}{rgb}{0.000000,0.000000,0.000000}%
\pgfsetstrokecolor{currentstroke}%
\pgfsetdash{}{0pt}%
\pgfsys@defobject{currentmarker}{\pgfqpoint{-0.048611in}{0.000000in}}{\pgfqpoint{-0.000000in}{0.000000in}}{%
\pgfpathmoveto{\pgfqpoint{-0.000000in}{0.000000in}}%
\pgfpathlineto{\pgfqpoint{-0.048611in}{0.000000in}}%
\pgfusepath{stroke,fill}%
}%
\begin{pgfscope}%
\pgfsys@transformshift{1.081443in}{2.415479in}%
\pgfsys@useobject{currentmarker}{}%
\end{pgfscope}%
\end{pgfscope}%
\begin{pgfscope}%
\definecolor{textcolor}{rgb}{0.000000,0.000000,0.000000}%
\pgfsetstrokecolor{textcolor}%
\pgfsetfillcolor{textcolor}%
\pgftext[x=0.775887in, y=2.367253in, left, base]{\color{textcolor}\rmfamily\fontsize{10.000000}{12.000000}\selectfont \(\displaystyle {520}\)}%
\end{pgfscope}%
\begin{pgfscope}%
\pgfpathrectangle{\pgfqpoint{1.081443in}{0.626234in}}{\pgfqpoint{5.175441in}{3.949094in}}%
\pgfusepath{clip}%
\pgfsetrectcap%
\pgfsetroundjoin%
\pgfsetlinewidth{0.803000pt}%
\definecolor{currentstroke}{rgb}{0.690196,0.690196,0.690196}%
\pgfsetstrokecolor{currentstroke}%
\pgfsetdash{}{0pt}%
\pgfpathmoveto{\pgfqpoint{1.081443in}{3.220349in}}%
\pgfpathlineto{\pgfqpoint{6.256884in}{3.220349in}}%
\pgfusepath{stroke}%
\end{pgfscope}%
\begin{pgfscope}%
\pgfsetbuttcap%
\pgfsetroundjoin%
\definecolor{currentfill}{rgb}{0.000000,0.000000,0.000000}%
\pgfsetfillcolor{currentfill}%
\pgfsetlinewidth{0.803000pt}%
\definecolor{currentstroke}{rgb}{0.000000,0.000000,0.000000}%
\pgfsetstrokecolor{currentstroke}%
\pgfsetdash{}{0pt}%
\pgfsys@defobject{currentmarker}{\pgfqpoint{-0.048611in}{0.000000in}}{\pgfqpoint{-0.000000in}{0.000000in}}{%
\pgfpathmoveto{\pgfqpoint{-0.000000in}{0.000000in}}%
\pgfpathlineto{\pgfqpoint{-0.048611in}{0.000000in}}%
\pgfusepath{stroke,fill}%
}%
\begin{pgfscope}%
\pgfsys@transformshift{1.081443in}{3.220349in}%
\pgfsys@useobject{currentmarker}{}%
\end{pgfscope}%
\end{pgfscope}%
\begin{pgfscope}%
\definecolor{textcolor}{rgb}{0.000000,0.000000,0.000000}%
\pgfsetstrokecolor{textcolor}%
\pgfsetfillcolor{textcolor}%
\pgftext[x=0.775887in, y=3.172124in, left, base]{\color{textcolor}\rmfamily\fontsize{10.000000}{12.000000}\selectfont \(\displaystyle {530}\)}%
\end{pgfscope}%
\begin{pgfscope}%
\pgfpathrectangle{\pgfqpoint{1.081443in}{0.626234in}}{\pgfqpoint{5.175441in}{3.949094in}}%
\pgfusepath{clip}%
\pgfsetrectcap%
\pgfsetroundjoin%
\pgfsetlinewidth{0.803000pt}%
\definecolor{currentstroke}{rgb}{0.690196,0.690196,0.690196}%
\pgfsetstrokecolor{currentstroke}%
\pgfsetdash{}{0pt}%
\pgfpathmoveto{\pgfqpoint{1.081443in}{4.025219in}}%
\pgfpathlineto{\pgfqpoint{6.256884in}{4.025219in}}%
\pgfusepath{stroke}%
\end{pgfscope}%
\begin{pgfscope}%
\pgfsetbuttcap%
\pgfsetroundjoin%
\definecolor{currentfill}{rgb}{0.000000,0.000000,0.000000}%
\pgfsetfillcolor{currentfill}%
\pgfsetlinewidth{0.803000pt}%
\definecolor{currentstroke}{rgb}{0.000000,0.000000,0.000000}%
\pgfsetstrokecolor{currentstroke}%
\pgfsetdash{}{0pt}%
\pgfsys@defobject{currentmarker}{\pgfqpoint{-0.048611in}{0.000000in}}{\pgfqpoint{-0.000000in}{0.000000in}}{%
\pgfpathmoveto{\pgfqpoint{-0.000000in}{0.000000in}}%
\pgfpathlineto{\pgfqpoint{-0.048611in}{0.000000in}}%
\pgfusepath{stroke,fill}%
}%
\begin{pgfscope}%
\pgfsys@transformshift{1.081443in}{4.025219in}%
\pgfsys@useobject{currentmarker}{}%
\end{pgfscope}%
\end{pgfscope}%
\begin{pgfscope}%
\definecolor{textcolor}{rgb}{0.000000,0.000000,0.000000}%
\pgfsetstrokecolor{textcolor}%
\pgfsetfillcolor{textcolor}%
\pgftext[x=0.775887in, y=3.976994in, left, base]{\color{textcolor}\rmfamily\fontsize{10.000000}{12.000000}\selectfont \(\displaystyle {540}\)}%
\end{pgfscope}%
\begin{pgfscope}%
\definecolor{textcolor}{rgb}{0.000000,0.000000,0.000000}%
\pgfsetstrokecolor{textcolor}%
\pgfsetfillcolor{textcolor}%
\pgftext[x=0.498109in,y=2.600781in,,bottom]{\color{textcolor}\rmfamily\fontsize{15.000000}{18.000000}\selectfont T in [K]}%
\end{pgfscope}%
\begin{pgfscope}%
\pgfpathrectangle{\pgfqpoint{1.081443in}{0.626234in}}{\pgfqpoint{5.175441in}{3.949094in}}%
\pgfusepath{clip}%
\pgfsetrectcap%
\pgfsetroundjoin%
\pgfsetlinewidth{1.505625pt}%
\definecolor{currentstroke}{rgb}{0.121569,0.466667,0.705882}%
\pgfsetstrokecolor{currentstroke}%
\pgfsetdash{}{0pt}%
\pgfpathmoveto{\pgfqpoint{1.085493in}{0.808153in}}%
\pgfpathlineto{\pgfqpoint{1.107451in}{0.818616in}}%
\pgfpathlineto{\pgfqpoint{1.129408in}{0.829885in}}%
\pgfpathlineto{\pgfqpoint{1.151366in}{0.840348in}}%
\pgfpathlineto{\pgfqpoint{1.173324in}{0.851616in}}%
\pgfpathlineto{\pgfqpoint{1.187962in}{0.858860in}}%
\pgfpathlineto{\pgfqpoint{1.239196in}{0.886225in}}%
\pgfpathlineto{\pgfqpoint{1.253835in}{0.893469in}}%
\pgfpathlineto{\pgfqpoint{1.356303in}{0.949005in}}%
\pgfpathlineto{\pgfqpoint{1.370941in}{0.956249in}}%
\pgfpathlineto{\pgfqpoint{1.422176in}{0.985224in}}%
\pgfpathlineto{\pgfqpoint{1.436814in}{0.994078in}}%
\pgfpathlineto{\pgfqpoint{1.488048in}{1.023053in}}%
\pgfpathlineto{\pgfqpoint{1.510006in}{1.036736in}}%
\pgfpathlineto{\pgfqpoint{1.531963in}{1.049614in}}%
\pgfpathlineto{\pgfqpoint{1.546602in}{1.058468in}}%
\pgfpathlineto{\pgfqpoint{1.568559in}{1.071346in}}%
\pgfpathlineto{\pgfqpoint{1.597836in}{1.089858in}}%
\pgfpathlineto{\pgfqpoint{1.612474in}{1.098711in}}%
\pgfpathlineto{\pgfqpoint{1.649070in}{1.122052in}}%
\pgfpathlineto{\pgfqpoint{1.663709in}{1.130906in}}%
\pgfpathlineto{\pgfqpoint{1.722262in}{1.168735in}}%
\pgfpathlineto{\pgfqpoint{1.736900in}{1.177588in}}%
\pgfpathlineto{\pgfqpoint{1.802773in}{1.221856in}}%
\pgfpathlineto{\pgfqpoint{1.817411in}{1.232320in}}%
\pgfpathlineto{\pgfqpoint{1.846688in}{1.252441in}}%
\pgfpathlineto{\pgfqpoint{1.868646in}{1.268539in}}%
\pgfpathlineto{\pgfqpoint{1.905242in}{1.293490in}}%
\pgfpathlineto{\pgfqpoint{1.963795in}{1.337758in}}%
\pgfpathlineto{\pgfqpoint{1.978433in}{1.348221in}}%
\pgfpathlineto{\pgfqpoint{2.029668in}{1.386855in}}%
\pgfpathlineto{\pgfqpoint{2.036987in}{1.391684in}}%
\pgfpathlineto{\pgfqpoint{2.051625in}{1.403757in}}%
\pgfpathlineto{\pgfqpoint{2.080902in}{1.427098in}}%
\pgfpathlineto{\pgfqpoint{2.095540in}{1.439171in}}%
\pgfpathlineto{\pgfqpoint{2.124817in}{1.462512in}}%
\pgfpathlineto{\pgfqpoint{2.146774in}{1.481024in}}%
\pgfpathlineto{\pgfqpoint{2.168732in}{1.498732in}}%
\pgfpathlineto{\pgfqpoint{2.293158in}{1.608999in}}%
\pgfpathlineto{\pgfqpoint{2.351711in}{1.666145in}}%
\pgfpathlineto{\pgfqpoint{2.366350in}{1.679827in}}%
\pgfpathlineto{\pgfqpoint{2.439542in}{1.753071in}}%
\pgfpathlineto{\pgfqpoint{2.498095in}{1.815850in}}%
\pgfpathlineto{\pgfqpoint{2.520053in}{1.838387in}}%
\pgfpathlineto{\pgfqpoint{2.556648in}{1.881045in}}%
\pgfpathlineto{\pgfqpoint{2.607883in}{1.938996in}}%
\pgfpathlineto{\pgfqpoint{2.644479in}{1.985678in}}%
\pgfpathlineto{\pgfqpoint{2.673755in}{2.022702in}}%
\pgfpathlineto{\pgfqpoint{2.761586in}{2.140213in}}%
\pgfpathlineto{\pgfqpoint{2.842096in}{2.256919in}}%
\pgfpathlineto{\pgfqpoint{2.886012in}{2.327748in}}%
\pgfpathlineto{\pgfqpoint{2.907969in}{2.362357in}}%
\pgfpathlineto{\pgfqpoint{2.937246in}{2.415479in}}%
\pgfpathlineto{\pgfqpoint{2.966523in}{2.465381in}}%
\pgfpathlineto{\pgfqpoint{2.981161in}{2.491137in}}%
\pgfpathlineto{\pgfqpoint{3.068991in}{2.667403in}}%
\pgfpathlineto{\pgfqpoint{3.127544in}{2.790548in}}%
\pgfpathlineto{\pgfqpoint{3.230013in}{3.062594in}}%
\pgfpathlineto{\pgfqpoint{3.244651in}{3.108472in}}%
\pgfpathlineto{\pgfqpoint{3.332482in}{3.395811in}}%
\pgfpathlineto{\pgfqpoint{3.354439in}{3.491590in}}%
\pgfpathlineto{\pgfqpoint{3.420312in}{3.778929in}}%
\pgfpathlineto{\pgfqpoint{3.508142in}{4.112950in}}%
\pgfpathlineto{\pgfqpoint{3.530099in}{4.170901in}}%
\pgfpathlineto{\pgfqpoint{3.588653in}{4.295655in}}%
\pgfpathlineto{\pgfqpoint{3.595972in}{4.311753in}}%
\pgfpathlineto{\pgfqpoint{3.603291in}{4.323021in}}%
\pgfpathlineto{\pgfqpoint{3.610610in}{4.331875in}}%
\pgfpathlineto{\pgfqpoint{3.639887in}{4.347167in}}%
\pgfpathlineto{\pgfqpoint{3.661845in}{4.357630in}}%
\pgfpathlineto{\pgfqpoint{3.669164in}{4.361655in}}%
\pgfpathlineto{\pgfqpoint{3.676483in}{4.360850in}}%
\pgfpathlineto{\pgfqpoint{3.705760in}{4.334289in}}%
\pgfpathlineto{\pgfqpoint{3.727717in}{4.313363in}}%
\pgfpathlineto{\pgfqpoint{3.742356in}{4.299680in}}%
\pgfpathlineto{\pgfqpoint{3.749675in}{4.290826in}}%
\pgfpathlineto{\pgfqpoint{3.800909in}{4.191022in}}%
\pgfpathlineto{\pgfqpoint{3.830186in}{4.112950in}}%
\pgfpathlineto{\pgfqpoint{3.837505in}{4.088804in}}%
\pgfpathlineto{\pgfqpoint{3.932654in}{3.719368in}}%
\pgfpathlineto{\pgfqpoint{3.998527in}{3.426396in}}%
\pgfpathlineto{\pgfqpoint{4.005846in}{3.395811in}}%
\pgfpathlineto{\pgfqpoint{4.049761in}{3.252544in}}%
\pgfpathlineto{\pgfqpoint{4.108315in}{3.062594in}}%
\pgfpathlineto{\pgfqpoint{4.137591in}{2.986132in}}%
\pgfpathlineto{\pgfqpoint{4.181506in}{2.868621in}}%
\pgfpathlineto{\pgfqpoint{4.210783in}{2.790548in}}%
\pgfpathlineto{\pgfqpoint{4.320571in}{2.563575in}}%
\pgfpathlineto{\pgfqpoint{4.364486in}{2.477454in}}%
\pgfpathlineto{\pgfqpoint{4.401082in}{2.415479in}}%
\pgfpathlineto{\pgfqpoint{4.444997in}{2.339016in}}%
\pgfpathlineto{\pgfqpoint{4.474274in}{2.292334in}}%
\pgfpathlineto{\pgfqpoint{4.503550in}{2.244846in}}%
\pgfpathlineto{\pgfqpoint{4.598700in}{2.110433in}}%
\pgfpathlineto{\pgfqpoint{4.715806in}{1.957508in}}%
\pgfpathlineto{\pgfqpoint{4.759722in}{1.905996in}}%
\pgfpathlineto{\pgfqpoint{4.803637in}{1.856094in}}%
\pgfpathlineto{\pgfqpoint{4.825594in}{1.831143in}}%
\pgfpathlineto{\pgfqpoint{4.884148in}{1.768363in}}%
\pgfpathlineto{\pgfqpoint{4.898786in}{1.753071in}}%
\pgfpathlineto{\pgfqpoint{4.928063in}{1.722486in}}%
\pgfpathlineto{\pgfqpoint{4.979297in}{1.672584in}}%
\pgfpathlineto{\pgfqpoint{4.993935in}{1.658901in}}%
\pgfpathlineto{\pgfqpoint{5.045170in}{1.608999in}}%
\pgfpathlineto{\pgfqpoint{5.198872in}{1.474586in}}%
\pgfpathlineto{\pgfqpoint{5.228149in}{1.451244in}}%
\pgfpathlineto{\pgfqpoint{5.242787in}{1.439171in}}%
\pgfpathlineto{\pgfqpoint{5.272064in}{1.415830in}}%
\pgfpathlineto{\pgfqpoint{5.286702in}{1.403757in}}%
\pgfpathlineto{\pgfqpoint{5.367213in}{1.342587in}}%
\pgfpathlineto{\pgfqpoint{5.381852in}{1.332124in}}%
\pgfpathlineto{\pgfqpoint{5.433086in}{1.293490in}}%
\pgfpathlineto{\pgfqpoint{5.469682in}{1.268539in}}%
\pgfpathlineto{\pgfqpoint{5.491639in}{1.252441in}}%
\pgfpathlineto{\pgfqpoint{5.528235in}{1.227490in}}%
\pgfpathlineto{\pgfqpoint{5.550193in}{1.211393in}}%
\pgfpathlineto{\pgfqpoint{5.659981in}{1.139760in}}%
\pgfpathlineto{\pgfqpoint{5.674619in}{1.130906in}}%
\pgfpathlineto{\pgfqpoint{5.711215in}{1.107565in}}%
\pgfpathlineto{\pgfqpoint{5.725853in}{1.098711in}}%
\pgfpathlineto{\pgfqpoint{5.755130in}{1.080199in}}%
\pgfpathlineto{\pgfqpoint{5.777087in}{1.067321in}}%
\pgfpathlineto{\pgfqpoint{5.791726in}{1.058468in}}%
\pgfpathlineto{\pgfqpoint{5.813683in}{1.045590in}}%
\pgfpathlineto{\pgfqpoint{5.835641in}{1.031907in}}%
\pgfpathlineto{\pgfqpoint{5.872237in}{1.010980in}}%
\pgfpathlineto{\pgfqpoint{5.886875in}{1.002127in}}%
\pgfpathlineto{\pgfqpoint{5.938109in}{0.973151in}}%
\pgfpathlineto{\pgfqpoint{5.952748in}{0.964298in}}%
\pgfpathlineto{\pgfqpoint{6.062535in}{0.904737in}}%
\pgfpathlineto{\pgfqpoint{6.077174in}{0.897494in}}%
\pgfpathlineto{\pgfqpoint{6.128408in}{0.870128in}}%
\pgfpathlineto{\pgfqpoint{6.143046in}{0.862884in}}%
\pgfpathlineto{\pgfqpoint{6.172323in}{0.847592in}}%
\pgfpathlineto{\pgfqpoint{6.194281in}{0.837128in}}%
\pgfpathlineto{\pgfqpoint{6.216238in}{0.825860in}}%
\pgfpathlineto{\pgfqpoint{6.238196in}{0.815397in}}%
\pgfpathlineto{\pgfqpoint{6.252834in}{0.808153in}}%
\pgfpathlineto{\pgfqpoint{6.252834in}{0.808153in}}%
\pgfusepath{stroke}%
\end{pgfscope}%
\begin{pgfscope}%
\pgfpathrectangle{\pgfqpoint{1.081443in}{0.626234in}}{\pgfqpoint{5.175441in}{3.949094in}}%
\pgfusepath{clip}%
\pgfsetrectcap%
\pgfsetroundjoin%
\pgfsetlinewidth{1.505625pt}%
\definecolor{currentstroke}{rgb}{1.000000,0.498039,0.054902}%
\pgfsetstrokecolor{currentstroke}%
\pgfsetdash{}{0pt}%
\pgfpathmoveto{\pgfqpoint{1.081443in}{0.805738in}}%
\pgfpathlineto{\pgfqpoint{1.193750in}{0.862576in}}%
\pgfpathlineto{\pgfqpoint{1.301399in}{0.919530in}}%
\pgfpathlineto{\pgfqpoint{1.404908in}{0.976791in}}%
\pgfpathlineto{\pgfqpoint{1.504277in}{1.034279in}}%
\pgfpathlineto{\pgfqpoint{1.600022in}{1.092224in}}%
\pgfpathlineto{\pgfqpoint{1.692145in}{1.150565in}}%
\pgfpathlineto{\pgfqpoint{1.780645in}{1.209231in}}%
\pgfpathlineto{\pgfqpoint{1.866040in}{1.268505in}}%
\pgfpathlineto{\pgfqpoint{1.947812in}{1.327957in}}%
\pgfpathlineto{\pgfqpoint{2.026479in}{1.387879in}}%
\pgfpathlineto{\pgfqpoint{2.102558in}{1.448624in}}%
\pgfpathlineto{\pgfqpoint{2.175531in}{1.509728in}}%
\pgfpathlineto{\pgfqpoint{2.245917in}{1.571562in}}%
\pgfpathlineto{\pgfqpoint{2.313716in}{1.634085in}}%
\pgfpathlineto{\pgfqpoint{2.378926in}{1.697245in}}%
\pgfpathlineto{\pgfqpoint{2.441549in}{1.760979in}}%
\pgfpathlineto{\pgfqpoint{2.502102in}{1.825776in}}%
\pgfpathlineto{\pgfqpoint{2.560584in}{1.891632in}}%
\pgfpathlineto{\pgfqpoint{2.616479in}{1.957904in}}%
\pgfpathlineto{\pgfqpoint{2.670304in}{2.025136in}}%
\pgfpathlineto{\pgfqpoint{2.722576in}{2.093990in}}%
\pgfpathlineto{\pgfqpoint{2.772777in}{2.163795in}}%
\pgfpathlineto{\pgfqpoint{2.821426in}{2.235275in}}%
\pgfpathlineto{\pgfqpoint{2.868005in}{2.307667in}}%
\pgfpathlineto{\pgfqpoint{2.913032in}{2.381763in}}%
\pgfpathlineto{\pgfqpoint{2.956505in}{2.457615in}}%
\pgfpathlineto{\pgfqpoint{2.998427in}{2.535275in}}%
\pgfpathlineto{\pgfqpoint{3.038795in}{2.614789in}}%
\pgfpathlineto{\pgfqpoint{3.077611in}{2.696201in}}%
\pgfpathlineto{\pgfqpoint{3.114874in}{2.779547in}}%
\pgfpathlineto{\pgfqpoint{3.150585in}{2.864854in}}%
\pgfpathlineto{\pgfqpoint{3.185260in}{2.953507in}}%
\pgfpathlineto{\pgfqpoint{3.218383in}{3.044335in}}%
\pgfpathlineto{\pgfqpoint{3.249953in}{3.137345in}}%
\pgfpathlineto{\pgfqpoint{3.280488in}{3.234225in}}%
\pgfpathlineto{\pgfqpoint{3.309988in}{3.335338in}}%
\pgfpathlineto{\pgfqpoint{3.337936in}{3.439103in}}%
\pgfpathlineto{\pgfqpoint{3.364848in}{3.547656in}}%
\pgfpathlineto{\pgfqpoint{3.390207in}{3.659116in}}%
\pgfpathlineto{\pgfqpoint{3.416085in}{3.783200in}}%
\pgfpathlineto{\pgfqpoint{3.439892in}{3.893037in}}%
\pgfpathlineto{\pgfqpoint{3.462664in}{3.987954in}}%
\pgfpathlineto{\pgfqpoint{3.484401in}{4.069302in}}%
\pgfpathlineto{\pgfqpoint{3.505102in}{4.138373in}}%
\pgfpathlineto{\pgfqpoint{3.524769in}{4.196397in}}%
\pgfpathlineto{\pgfqpoint{3.543401in}{4.244541in}}%
\pgfpathlineto{\pgfqpoint{3.560997in}{4.283914in}}%
\pgfpathlineto{\pgfqpoint{3.577558in}{4.315559in}}%
\pgfpathlineto{\pgfqpoint{3.593085in}{4.340462in}}%
\pgfpathlineto{\pgfqpoint{3.607576in}{4.359544in}}%
\pgfpathlineto{\pgfqpoint{3.621032in}{4.373665in}}%
\pgfpathlineto{\pgfqpoint{3.633453in}{4.383626in}}%
\pgfpathlineto{\pgfqpoint{3.645357in}{4.390403in}}%
\pgfpathlineto{\pgfqpoint{3.656743in}{4.394348in}}%
\pgfpathlineto{\pgfqpoint{3.667611in}{4.395801in}}%
\pgfpathlineto{\pgfqpoint{3.678480in}{4.394994in}}%
\pgfpathlineto{\pgfqpoint{3.689348in}{4.391927in}}%
\pgfpathlineto{\pgfqpoint{3.700734in}{4.386291in}}%
\pgfpathlineto{\pgfqpoint{3.712637in}{4.377746in}}%
\pgfpathlineto{\pgfqpoint{3.725059in}{4.365941in}}%
\pgfpathlineto{\pgfqpoint{3.737997in}{4.350505in}}%
\pgfpathlineto{\pgfqpoint{3.751971in}{4.330237in}}%
\pgfpathlineto{\pgfqpoint{3.766980in}{4.304307in}}%
\pgfpathlineto{\pgfqpoint{3.782506in}{4.272949in}}%
\pgfpathlineto{\pgfqpoint{3.799067in}{4.234416in}}%
\pgfpathlineto{\pgfqpoint{3.816664in}{4.187727in}}%
\pgfpathlineto{\pgfqpoint{3.835295in}{4.131835in}}%
\pgfpathlineto{\pgfqpoint{3.854962in}{4.065633in}}%
\pgfpathlineto{\pgfqpoint{3.875664in}{3.987954in}}%
\pgfpathlineto{\pgfqpoint{3.897401in}{3.897567in}}%
\pgfpathlineto{\pgfqpoint{3.920173in}{3.793182in}}%
\pgfpathlineto{\pgfqpoint{3.963129in}{3.591985in}}%
\pgfpathlineto{\pgfqpoint{3.990041in}{3.479773in}}%
\pgfpathlineto{\pgfqpoint{4.017989in}{3.372796in}}%
\pgfpathlineto{\pgfqpoint{4.046971in}{3.270555in}}%
\pgfpathlineto{\pgfqpoint{4.077506in}{3.170994in}}%
\pgfpathlineto{\pgfqpoint{4.109076in}{3.075599in}}%
\pgfpathlineto{\pgfqpoint{4.142199in}{2.982606in}}%
\pgfpathlineto{\pgfqpoint{4.176357in}{2.893293in}}%
\pgfpathlineto{\pgfqpoint{4.212068in}{2.806134in}}%
\pgfpathlineto{\pgfqpoint{4.249331in}{2.721097in}}%
\pgfpathlineto{\pgfqpoint{4.287629in}{2.639210in}}%
\pgfpathlineto{\pgfqpoint{4.327480in}{2.559219in}}%
\pgfpathlineto{\pgfqpoint{4.368883in}{2.481086in}}%
\pgfpathlineto{\pgfqpoint{4.411840in}{2.404765in}}%
\pgfpathlineto{\pgfqpoint{4.456348in}{2.330207in}}%
\pgfpathlineto{\pgfqpoint{4.502410in}{2.257362in}}%
\pgfpathlineto{\pgfqpoint{4.550541in}{2.185425in}}%
\pgfpathlineto{\pgfqpoint{4.600226in}{2.115176in}}%
\pgfpathlineto{\pgfqpoint{4.651980in}{2.045879in}}%
\pgfpathlineto{\pgfqpoint{4.705287in}{1.978218in}}%
\pgfpathlineto{\pgfqpoint{4.760664in}{1.911520in}}%
\pgfpathlineto{\pgfqpoint{4.818112in}{1.845814in}}%
\pgfpathlineto{\pgfqpoint{4.877629in}{1.781118in}}%
\pgfpathlineto{\pgfqpoint{4.939217in}{1.717443in}}%
\pgfpathlineto{\pgfqpoint{5.003393in}{1.654297in}}%
\pgfpathlineto{\pgfqpoint{5.070156in}{1.591750in}}%
\pgfpathlineto{\pgfqpoint{5.138989in}{1.530310in}}%
\pgfpathlineto{\pgfqpoint{5.210410in}{1.469530in}}%
\pgfpathlineto{\pgfqpoint{5.284937in}{1.409039in}}%
\pgfpathlineto{\pgfqpoint{5.362051in}{1.349317in}}%
\pgfpathlineto{\pgfqpoint{5.442270in}{1.290010in}}%
\pgfpathlineto{\pgfqpoint{5.525595in}{1.231183in}}%
\pgfpathlineto{\pgfqpoint{5.612024in}{1.172891in}}%
\pgfpathlineto{\pgfqpoint{5.702077in}{1.114851in}}%
\pgfpathlineto{\pgfqpoint{5.795753in}{1.057143in}}%
\pgfpathlineto{\pgfqpoint{5.893051in}{0.999835in}}%
\pgfpathlineto{\pgfqpoint{5.993972in}{0.942984in}}%
\pgfpathlineto{\pgfqpoint{6.099034in}{0.886364in}}%
\pgfpathlineto{\pgfqpoint{6.208235in}{0.830050in}}%
\pgfpathlineto{\pgfqpoint{6.256367in}{0.805995in}}%
\pgfpathlineto{\pgfqpoint{6.256367in}{0.805995in}}%
\pgfusepath{stroke}%
\end{pgfscope}%
\begin{pgfscope}%
\pgfsetrectcap%
\pgfsetmiterjoin%
\pgfsetlinewidth{0.803000pt}%
\definecolor{currentstroke}{rgb}{0.000000,0.000000,0.000000}%
\pgfsetstrokecolor{currentstroke}%
\pgfsetdash{}{0pt}%
\pgfpathmoveto{\pgfqpoint{1.081443in}{0.626234in}}%
\pgfpathlineto{\pgfqpoint{1.081443in}{4.575328in}}%
\pgfusepath{stroke}%
\end{pgfscope}%
\begin{pgfscope}%
\pgfsetrectcap%
\pgfsetmiterjoin%
\pgfsetlinewidth{0.803000pt}%
\definecolor{currentstroke}{rgb}{0.000000,0.000000,0.000000}%
\pgfsetstrokecolor{currentstroke}%
\pgfsetdash{}{0pt}%
\pgfpathmoveto{\pgfqpoint{6.256884in}{0.626234in}}%
\pgfpathlineto{\pgfqpoint{6.256884in}{4.575328in}}%
\pgfusepath{stroke}%
\end{pgfscope}%
\begin{pgfscope}%
\pgfsetrectcap%
\pgfsetmiterjoin%
\pgfsetlinewidth{0.803000pt}%
\definecolor{currentstroke}{rgb}{0.000000,0.000000,0.000000}%
\pgfsetstrokecolor{currentstroke}%
\pgfsetdash{}{0pt}%
\pgfpathmoveto{\pgfqpoint{1.081443in}{0.626234in}}%
\pgfpathlineto{\pgfqpoint{6.256884in}{0.626234in}}%
\pgfusepath{stroke}%
\end{pgfscope}%
\begin{pgfscope}%
\pgfsetrectcap%
\pgfsetmiterjoin%
\pgfsetlinewidth{0.803000pt}%
\definecolor{currentstroke}{rgb}{0.000000,0.000000,0.000000}%
\pgfsetstrokecolor{currentstroke}%
\pgfsetdash{}{0pt}%
\pgfpathmoveto{\pgfqpoint{1.081443in}{4.575328in}}%
\pgfpathlineto{\pgfqpoint{6.256884in}{4.575328in}}%
\pgfusepath{stroke}%
\end{pgfscope}%
\begin{pgfscope}%
\pgfsetbuttcap%
\pgfsetmiterjoin%
\definecolor{currentfill}{rgb}{1.000000,1.000000,1.000000}%
\pgfsetfillcolor{currentfill}%
\pgfsetlinewidth{1.003750pt}%
\definecolor{currentstroke}{rgb}{0.000000,0.000000,0.000000}%
\pgfsetstrokecolor{currentstroke}%
\pgfsetdash{}{0pt}%
\pgfpathmoveto{\pgfqpoint{4.764416in}{3.830884in}}%
\pgfpathlineto{\pgfqpoint{6.111051in}{3.830884in}}%
\pgfpathquadraticcurveto{\pgfqpoint{6.152718in}{3.830884in}}{\pgfqpoint{6.152718in}{3.872551in}}%
\pgfpathlineto{\pgfqpoint{6.152718in}{4.429495in}}%
\pgfpathquadraticcurveto{\pgfqpoint{6.152718in}{4.471161in}}{\pgfqpoint{6.111051in}{4.471161in}}%
\pgfpathlineto{\pgfqpoint{4.764416in}{4.471161in}}%
\pgfpathquadraticcurveto{\pgfqpoint{4.722750in}{4.471161in}}{\pgfqpoint{4.722750in}{4.429495in}}%
\pgfpathlineto{\pgfqpoint{4.722750in}{3.872551in}}%
\pgfpathquadraticcurveto{\pgfqpoint{4.722750in}{3.830884in}}{\pgfqpoint{4.764416in}{3.830884in}}%
\pgfpathclose%
\pgfusepath{stroke,fill}%
\end{pgfscope}%
\begin{pgfscope}%
\pgfsetrectcap%
\pgfsetroundjoin%
\pgfsetlinewidth{1.505625pt}%
\definecolor{currentstroke}{rgb}{0.121569,0.466667,0.705882}%
\pgfsetstrokecolor{currentstroke}%
\pgfsetdash{}{0pt}%
\pgfpathmoveto{\pgfqpoint{4.806083in}{4.314911in}}%
\pgfpathlineto{\pgfqpoint{5.222750in}{4.314911in}}%
\pgfusepath{stroke}%
\end{pgfscope}%
\begin{pgfscope}%
\definecolor{textcolor}{rgb}{0.000000,0.000000,0.000000}%
\pgfsetstrokecolor{textcolor}%
\pgfsetfillcolor{textcolor}%
\pgftext[x=5.389416in,y=4.241995in,left,base]{\color{textcolor}\rmfamily\fontsize{15.000000}{18.000000}\selectfont coarse}%
\end{pgfscope}%
\begin{pgfscope}%
\pgfsetrectcap%
\pgfsetroundjoin%
\pgfsetlinewidth{1.505625pt}%
\definecolor{currentstroke}{rgb}{1.000000,0.498039,0.054902}%
\pgfsetstrokecolor{currentstroke}%
\pgfsetdash{}{0pt}%
\pgfpathmoveto{\pgfqpoint{4.806083in}{4.026023in}}%
\pgfpathlineto{\pgfqpoint{5.222750in}{4.026023in}}%
\pgfusepath{stroke}%
\end{pgfscope}%
\begin{pgfscope}%
\definecolor{textcolor}{rgb}{0.000000,0.000000,0.000000}%
\pgfsetstrokecolor{textcolor}%
\pgfsetfillcolor{textcolor}%
\pgftext[x=5.389416in,y=3.953106in,left,base]{\color{textcolor}\rmfamily\fontsize{15.000000}{18.000000}\selectfont analytic}%
\end{pgfscope}%
\end{pgfpicture}%
\makeatother%
\endgroup%
}
	\caption{Comparison of the steady-state temperature distribution of the analytical solution (refEQ) and the numerical solution on the coarse mesh}
	\label{fig::TemperatureDist}
\end{figure}
