\begin{abstract}
	\textbf{\textsf{Abstract:}}	
	A finite element method based solver for the simulation of the unsteady heat equation on a two dimensional disk is first optimized for serial performance and then in a subsequent step parallelized. The different meshes were analysed. The performance analysis was conducted on Intel Platinum 8160 CPUs. The influence on the runtime of different compiler flag are tested and compared, where aggressive optimization with activation of the automatic vectorizer using AVX512 instruction set extensions showed the best performance gain. Parallelizations with OpenMP and MPI were assessed. For the OpenMP parallelization the two most time consuming loops where parallelized, achieving highest performance with reduction clauses, with static scheduling and default chunk size. One-sided communication is used for the MPI-parallelization, with a round-robin partitioning. Different numbers of threads / cores are compared for each parallelization approach, regarding runtime, speed-up and efficiency. The obtained performance gain with the MPI parallelization was poor, as for a higher number of cores, no speed-up compared to the serial runtime could be observed. Improvements of the partitioning and the one-sided communication pattern are discussed. 
	
	
\end{abstract}
\thispagestyle{empty}